\chapter{Finite Differenzenverfahren für die Poisson-Gleichung}


\begin{defi}
Sei $\Omega \subset \mathbb R^2$ ein \textbf{Gebiet}, d.h. eine offene, beschränkte und zusammenhängende Menge
mit Rand $\partial \Omega$, eine stückweise $C^1$-Kurve.

Für $u: \Omega \to \mathbb R$, $(x,y) \mapsto u(x,y)$, $u \in C^2(\Omega)$ heisst 
\[
\Delta u := \frac{\partial^2 u}{\partial x^2} + \frac{\partial^2 u}{\partial y^2}
\]
der \textbf{Laplace-Operator}.

Das \textbf{Poisson-Problem} mit Dirichlet-Randbedingungen lautet
\begin{align*}
- \Delta u &= f, \text{ in } \Omega \\
u &= g, \text{ auf } \partial \Omega
\end{align*}
Falls $f$ und $g$ stetig sind, sucht man $u \in C^2(\Omega) \cap C(\overline \Omega)$.
\end{defi}

\begin{bem}
\begin{enumerate}
\item Explizite Lösungsformeln existieren nur in einfachen Geometrien.
\item Andere Randbedingungen sind
\begin{itemize}
\item Neumann-Randbedingungen: $\frac{\partial u}{\partial n} = g$ auf $\partial \Omega$
\item Gemischte oder Robin-Randbedingungen: $\alpha u + \beta \frac{\partial u}{\partial n} = g$ auf $\partial \Omega$, 
$|\alpha| + |\beta| \neq 0$
\end{itemize}
\end{enumerate}
\end{bem}

\section{Finite Differenzen Diskretisierung in $\Omega = (0,1) \times (0,1)$}
Für $n \geq 1$ führe ein äquidistantes Gitter mit Maschenweite $h = \frac{1}{n+1}$ ein:
\begin{align*}
x_j &= j \cdot h, \; j = 0, \ldots, n+1 \\
y_j &= j \cdot h, \; j = 0, \ldots, n+1 \\
\Omega_h &= \{ (x_j, y_k) \ : \ 1 \leq j, k \leq n\} \\
\overline \Omega_h &= \{ (x_j, y_k) \ : \ 0 \leq j, k \leq n+1 \} \\
\partial \Omega_h &= \overline \Omega_h \setminus \Omega_h
\end{align*}

Für $u \in C^4(\Omega)$ gilt:
\begin{align*}
\frac{\partial^2 u}{\partial x^2}(x_i, y_j)
    &= \frac{1}{h^2} \left[ u(x_{i+1}, y_j) - 2u(x_i, y_j) + u(x_{i-1}, y_j) \right] + \mathcal O(h^2) \\
\frac{\partial^2 u}{\partial y^2}(x_i, y_j)
    &= \frac{1}{h^2} \left[ u(x_i, y_{j+1}) - 2u(x_i, y_j) + u(x_i, y_{j-1}) \right] + \mathcal O(h^2)
\end{align*}

Wir ersetzen das Poisson-Problem durch das diskrete Poisson-Problem:

Finde $\{u_{ij}\}_{1 \leq i, j \leq n}$ so, dass
\begin{align*}
 - \frac{1}{h^2} \left[ u_{i+1, j} + u_{i-1, j} + u_{i, j+1} + u_{i, j-1} - 4 u_{ij} \right] &= f_{ij}, \text{ für }
    (x_i, y_j) \in \Omega_h \\
 u_{ij} &= g_{ij}, \text{ für } (x_i, y_j) \in \partial \Omega_h
\end{align*}

Oder anders formuliert, finde $u_h: \overline \Omega_h \to \mathbb R$ so, dass
\begin{align*}
- \Delta_h u_h(z) &= f(z), \text{ für } z \in \Omega_h \\
u_h(z) &= g(z), \text{ für } z \in \partial \Omega_h ,
\end{align*}
wobei 
\[
( \Delta_h u_h)(x_i, y_j)
    = \frac{1}{h^2}
    \left[
    u_h(x_{i+1}, y_j) + u_h(x_{i-1}, y_j) + u_h(x_i, y_{j+1}) + u_h(x_i, y_{j-1}) - 4 u_h(x_i, y_j) 
    \right]
\]

\section{Diskretes Maximumprinzip}
Für $n \geq 1$, $h = \frac{1}{n+1}$, betrachte die Differenzengleichung
\[
L_h u_h(z) = f(x), \text{ für } z \in \Omega_h ,
\]
wobei 
\begin{align*}
L_h u_h(z) &= \sum_{z_k \in N(z)} \alpha_k (z_k) u_h(z_k), \\
N(z) &= \{ z_k \ : \ k = 0, 1, \ldots, 4 \}
\end{align*}

\begin{satz}[Diskretes Maximumsprinzip]
Sei $u_h: \overline \Omega_h \to \mathbb R$ Lösung der Differenzengleichung
\[
L_h u_h(z) = f(z), \ z \in \Omega_h
\]
mit $f(z) \leq 0$ für $z \in \Omega_h$.
Ausserdem gelte für $z \in \Omega_h$:
\begin{itemize}
\item[(i)] $ \alpha_k(z_k) < 0$ falls $1 \leq k \leq 4$
\item[(ii)] $\sum_{z_k \in N(z)} \alpha_k(z_k) = 0 $
\end{itemize}
Dann folgt
\[
\max_{z \in \Omega_h} u_h(z) \leq \max_{z \in \partial \Omega_h} u_h(z)
\]
\end{satz}
\begin{proof}
Angenommen, es gibt ein $\overline z \in \Omega_h$ mit $u_h(\overline z) = \max_{z \in \overline \Omega_h} u_h(z)$
und $u_h(\overline z) > \max_{z \in \partial \Omega_h} u_h(z)$.

Es gilt:
\begin{align*}
0 &\leq \sum_{z_k \in N(\overline z) \setminus \{ \overline z \}} \alpha_k(z_k) \left[ u_h(z_k) - u_h(\overline z) \right] \\
    &= \sum_{z_k \in N(\overline z)} \alpha_k (z_k) \left[ u_h(z_k) - u_h(\overline z) \right] \\
    &= \sum_{z_k \in N(\overline z)} \alpha_k (z_k) u_h(z_k) - u_h(\overline z) \sum_{z_k \in N(\overline z)} \alpha_k(z_k) \\
    &= L_h u_h(\overline z) - 0 = f(\overline z) \leq 0
\end{align*}
Also folgt $u_h(z_k) = u_h(\overline z)$ für $z_k \in N(\overline z)$.
Wiederhole den Beweis mit $\overline z \in N(\overline z)\setminus \{\overline z\}$, u.s.w.
Nach endlich vielen Schritten kommt man zum Rand und erhält einen Widerspruch.
\end{proof}

\begin{kor}
Seien die beiden Voraussetzungen des Satzes an $L_h$ erfüllt. Dann hat
\begin{align*}
L_h u_h( z) &= f(z), \text{ für } z \in \Omega_h \\
u_h(z) &= g(z), \text{ für } z \in \partial \Omega_h
\end{align*}
eine eindeutige Lösung.
\end{kor}
\begin{proof}
Angenommen es existieren zwei Lösungen $u_h$ und $v_h$. Dann erfüllt $w_h := u_h - v_h$
\begin{align*}
L_h w_h(z) &= 0, \text{ für } z \in \Omega_h \\
w_h(z) &= 0, \text{ für } z \in \partial \Omega_h
\end{align*}
Nach dem diskreten Maximumsprinzip folgt $w_h(z) \leq 0$ für $z \in \Omega_h$.

Ausserdem gilt:
\begin{align*}
L_h (-w_h(z)) &= 0, \text{ für } z \in \Omega_h \\
-w_h(z) &= 0, \text{ für } z \in \partial \Omega_h
\end{align*}
Also folgt wieder mit dem diskreten Maximumsprinzip $-w_h(z) \leq 0$ für $z \in \Omega_h$.
Damit ist aber $w_h = 0$ auf $\overline \Omega_h$, also $u_h = v_h$.
\end{proof}


