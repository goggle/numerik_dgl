\documentclass[a4paper,11pt,DIV0]{scrreprt}

\usepackage{ucs}
\usepackage[utf8x]{inputenc}
\usepackage[T1]{fontenc}
\usepackage{scrpage2}
\usepackage{amsmath, amssymb, amsthm}
%\usepackage{listings, color}
\usepackage{graphicx}
\usepackage[ngerman]{babel}
%\usepackage{moreverb}
%\usepackage{tabularx}
%\usepackage{bbold}

\KOMAoption{chapterprefix}{}

%\lstset{language=Matlab, numbers=left, numberstyle=\tiny, tabsize=4,%
%	keywordstyle=\color{blue}\bfseries, commentstyle=\color{gruen}\ttfamily,%
%	basicstyle=\footnotesize, identifierstyle=\ttfamily, %
%	stringstyle=\color{red}, showstringspaces=false, %
%	stepnumber=5, firstnumber=1, breaklines=true,%
%	numberfirstline=true, emphstyle=\bfseries, frame=single}
%\definecolor{gruen}{rgb}{0.25,0.6,0.1}

\renewcommand{\thechapter}{\Roman{chapter}}
\renewcommand{\thesection}{\arabic{section}}
\newcommand{\bbone}{1\hspace{-0,9ex}1} 

\newtheoremstyle{aufgabenstyle}%
	{}%spaceabove
	{}%spacebelow
	{\bfseries}%bodyfont
	{}%indent
	{\scshape}%headfont
	{.}%headpunctuation
	{}%headspace
	{}%headspec
	%{\thmname{#1}}%\thmnumber{ #2}}%headspec

\theoremstyle{definition}
\newtheorem{aufgabe}{Aufgabe}[section]
\newtheorem*{defi}{Definition}
\newtheorem*{lem}{Lemma}
\newtheorem{satz}{Satz}[section]
\newtheorem*{kor}{Korollar}
\newtheorem*{bem}{Bemerkung}
\newtheorem*{idee}{Idee}
\newtheorem*{bsp}{Beispiel}
\newtheorem*{bspe}{Beispiele}
\newtheorem*{konv}{Konvention}




%\ihead{}
%\ohead{}
%\setlength{\headheight}{4em}
%\setheadsepline{0.5pt}
%\pagestyle{scrheadings}


\parindent0pt

\makeatletter 
\renewenvironment{proof}[1][\proofname]{% 
% 
\setlength{\topsep}{0pt} 
\pushQED{\qed}% 
\normalfont %\topsep6\p@\@plus6\p@\relax 
\trivlist 
\item[\hskip\labelsep 
	\bfseries 
	#1\@addpunct{.}]\ignorespaces 
}{% 
	\popQED\endtrivlist\@endpefalse 
} 
\makeatother 

\DeclareMathOperator{\spann}{span}
\DeclareMathOperator{\sign}{sign}
\DeclareMathOperator{\pr}{pr}
\DeclareMathOperator{\Div}{div}
\DeclareMathOperator{\supp}{supp}



\begin{document}

\title{Zusammenfassung - Numerik der Differentialgleichungen}
\author{}
\date{}
\maketitle
\newpage


\chapter{Einleitung}

\subsection*{Verallgemeinerte partielle Integration}
Integralsatz von Gauss:
\[
\int_\Omega \Div F \mathrm dx \ \mathrm dy = \int_{\partial \Omega} \langle F, v \rangle \mathrm ds
\]

Sei $F(x,y) := v \nabla u = (u_x v, u_y v)$.
Dann ist 
\[
\Div F = \frac{\partial F_1}{\partial x} + \frac{\partial F_2}{\partial y}
    = u_{xx} v + u_x v_x + u_{yy} v + u_y v_y
    = v \Delta u + \langle \nabla u, \nabla v \rangle
\]

Einsetzen in den Integralsatz liefert nun:
\[
\int_\Omega \langle \nabla u, \nabla v \rangle \ \mathrm dx \ \mathrm dy
    = - \int_\Omega v \cdot \nabla u \ \mathrm dx \ \mathrm dy
    + \int_{\partial \Omega} v \langle \nabla u, v \rangle \ \mathrm ds
\]

\chapter{Finite Differenzenverfahren für die Poisson-Gleichung}


\begin{defi}
Sei $\Omega \subset \mathbb R^2$ ein \textbf{Gebiet}, d.h. eine offene, beschränkte und zusammenhängende Menge
mit Rand $\partial \Omega$, eine stückweise $C^1$-Kurve.

Für $u: \Omega \to \mathbb R$, $(x,y) \mapsto u(x,y)$, $u \in C^2(\Omega)$ heisst 
\[
\Delta u := \frac{\partial^2 u}{\partial x^2} + \frac{\partial^2 u}{\partial y^2}
\]
der \textbf{Laplace-Operator}.

Das \textbf{Poisson-Problem} mit Dirichlet-Randbedingungen lautet
\begin{align*}
- \Delta u &= f, \text{ in } \Omega \\
u &= g, \text{ auf } \partial \Omega
\end{align*}
Falls $f$ und $g$ stetig sind, sucht man $u \in C^2(\Omega) \cap C(\overline \Omega)$.
\end{defi}

\begin{bem}
\begin{enumerate}
\item Explizite Lösungsformeln existieren nur in einfachen Geometrien.
\item Andere Randbedingungen sind
\begin{itemize}
\item Neumann-Randbedingungen: $\frac{\partial u}{\partial n} = g$ auf $\partial \Omega$
\item Gemischte oder Robin-Randbedingungen: $\alpha u + \beta \frac{\partial u}{\partial n} = g$ auf $\partial \Omega$, 
$|\alpha| + |\beta| \neq 0$
\end{itemize}
\end{enumerate}
\end{bem}

\section{Finite Differenzen Diskretisierung in $\Omega = (0,1) \times (0,1)$}
Für $n \geq 1$ führe ein äquidistantes Gitter mit Maschenweite $h = \frac{1}{n+1}$ ein:
\begin{align*}
x_j &= j \cdot h, \; j = 0, \ldots, n+1 \\
y_j &= j \cdot h, \; j = 0, \ldots, n+1 \\
\Omega_h &= \{ (x_j, y_k) \ : \ 1 \leq j, k \leq n\} \\
\overline \Omega_h &= \{ (x_j, y_k) \ : \ 0 \leq j, k \leq n+1 \} \\
\partial \Omega_h &= \overline \Omega_h \setminus \Omega_h
\end{align*}

Für $u \in C^4(\Omega)$ gilt:
\begin{align*}
\frac{\partial^2 u}{\partial x^2}(x_i, y_j)
    &= \frac{1}{h^2} \left[ u(x_{i+1}, y_j) - 2u(x_i, y_j) + u(x_{i-1}, y_j) \right] + \mathcal O(h^2) \\
\frac{\partial^2 u}{\partial y^2}(x_i, y_j)
    &= \frac{1}{h^2} \left[ u(x_i, y_{j+1}) - 2u(x_i, y_j) + u(x_i, y_{j-1}) \right] + \mathcal O(h^2)
\end{align*}

Wir ersetzen das Poisson-Problem durch das diskrete Poisson-Problem:

Finde $\{u_{ij}\}_{1 \leq i, j \leq n}$ so, dass
\begin{align*}
 - \frac{1}{h^2} \left[ u_{i+1, j} + u_{i-1, j} + u_{i, j+1} + u_{i, j-1} - 4 u_{ij} \right] &= f_{ij}, \text{ für }
    (x_i, y_j) \in \Omega_h \\
 u_{ij} &= g_{ij}, \text{ für } (x_i, y_j) \in \partial \Omega_h
\end{align*}

Oder anders formuliert, finde $u_h: \overline \Omega_h \to \mathbb R$ so, dass
\begin{align*}
- \Delta_h u_h(z) &= f(z), \text{ für } z \in \Omega_h \\
u_h(z) &= g(z), \text{ für } z \in \partial \Omega_h ,
\end{align*}
wobei 
\[
( \Delta_h u_h)(x_i, y_j)
    = \frac{1}{h^2}
    \left[
    u_h(x_{i+1}, y_j) + u_h(x_{i-1}, y_j) + u_h(x_i, y_{j+1}) + u_h(x_i, y_{j-1}) - 4 u_h(x_i, y_j) 
    \right]
\]

\section{Diskretes Maximumprinzip}
Für $n \geq 1$, $h = \frac{1}{n+1}$, betrachte die Differenzengleichung
\[
L_h u_h(z) = f(x), \text{ für } z \in \Omega_h ,
\]
wobei 
\begin{align*}
L_h u_h(z) &= \sum_{z_k \in N(z)} \alpha_k (z_k) u_h(z_k), \\
N(z) &= \{ z_k \ : \ k = 0, 1, \ldots, 4 \}
\end{align*}

\begin{satz}[Diskretes Maximumsprinzip]
Sei $u_h: \overline \Omega_h \to \mathbb R$ Lösung der Differenzengleichung
\[
L_h u_h(z) = f(z), \ z \in \Omega_h
\]
mit $f(z) \leq 0$ für $z \in \Omega_h$.
Ausserdem gelte für $z \in \Omega_h$:
\begin{itemize}
\item[(i)] $ \alpha_k(z_k) < 0$ falls $1 \leq k \leq 4$
\item[(ii)] $\sum_{z_k \in N(z)} \alpha_k(z_k) = 0 $
\end{itemize}
Dann folgt
\[
\max_{z \in \Omega_h} u_h(z) \leq \max_{z \in \partial \Omega_h} u_h(z)
\]
\end{satz}
\begin{proof}
Angenommen, es gibt ein $\overline z \in \Omega_h$ mit $u_h(\overline z) = \max_{z \in \overline \Omega_h} u_h(z)$
und $u_h(\overline z) > \max_{z \in \partial \Omega_h} u_h(z)$.

Es gilt:
\begin{align*}
0 &\leq \sum_{z_k \in N(\overline z) \setminus \{ \overline z \}} \alpha_k(z_k) \left[ u_h(z_k) - u_h(\overline z) \right] \\
    &= \sum_{z_k \in N(\overline z)} \alpha_k (z_k) \left[ u_h(z_k) - u_h(\overline z) \right] \\
    &= \sum_{z_k \in N(\overline z)} \alpha_k (z_k) u_h(z_k) - u_h(\overline z) \sum_{z_k \in N(\overline z)} \alpha_k(z_k) \\
    &= L_h u_h(\overline z) - 0 = f(\overline z) \leq 0
\end{align*}
Also folgt $u_h(z_k) = u_h(\overline z)$ für $z_k \in N(\overline z)$.
Wiederhole den Beweis mit $\overline z \in N(\overline z)\setminus \{\overline z\}$, u.s.w.
Nach endlich vielen Schritten kommt man zum Rand und erhält einen Widerspruch.
\end{proof}

\begin{kor}
Seien die beiden Voraussetzungen des Satzes an $L_h$ erfüllt. Dann hat
\begin{align*}
L_h u_h( z) &= f(z), \text{ für } z \in \Omega_h \\
u_h(z) &= g(z), \text{ für } z \in \partial \Omega_h
\end{align*}
eine eindeutige Lösung.
\end{kor}
\begin{proof}
Angenommen es existieren zwei Lösungen $u_h$ und $v_h$. Dann erfüllt $w_h := u_h - v_h$
\begin{align*}
L_h w_h(z) &= 0, \text{ für } z \in \Omega_h \\
w_h(z) &= 0, \text{ für } z \in \partial \Omega_h
\end{align*}
Nach dem diskreten Maximumsprinzip folgt $w_h(z) \leq 0$ für $z \in \Omega_h$.

Ausserdem gilt:
\begin{align*}
L_h (-w_h(z)) &= 0, \text{ für } z \in \Omega_h \\
-w_h(z) &= 0, \text{ für } z \in \partial \Omega_h
\end{align*}
Also folgt wieder mit dem diskreten Maximumsprinzip $-w_h(z) \leq 0$ für $z \in \Omega_h$.
Damit ist aber $w_h = 0$ auf $\overline \Omega_h$, also $u_h = v_h$.
\end{proof}

\section{Konvergenz der Finiten Differenzen Methode}
Wir betrachten das Poisson-Problem
\begin{align*}
- \Delta u &= f \text{ in } \Omega \\
u &= g \text{ auf } \partial \Omega
\end{align*}
und das entsprechende diskrete Problem
\begin{align*}
- \Delta_h u_h &= f \text{ in } \Omega_h \\
u_h &= g \text{ auf } \partial \Omega_h
\end{align*}
und fragen uns, ob das Finite Differenzen Verfahren konvergiert, d.h. ob
\[
|u(z) - u_h(z)| \to 0 \text{ für } h \to 0
\]

\begin{defi}
Ein Finite Differenzen Verfahren $L_h$ heisst \textbf{konsistent} mit einem partiellen
Differentialoperator (2. Ordnung) $L$, falls 
\[
Lu(z) - L_h u_h(z) \to 0 \text{ für } h \to 0 \text{ und für alle } z \in \Omega_h
\]

Ein Finite Differenzenverfahren hat die \textbf{Konsistenzordnung} $m$, falls 
\[
L u(z) - L_h u_h(z) = \mathcal O(h^m) \text{ für } h \to 0
\]
und für alle $z \in \Omega_h$, $u: \Omega \to \mathbb R$ genügend oft differenzierbar.
\end{defi}

\begin{bem}
Sei $\eta_h (z) := u(z) - u_h(z)$, $z \in \Omega_h$ der Fehler.
Dann gilt für $z \in \Omega_h$:
\begin{align*}
L_h \eta_h (z)
    &= L_h (u - u_h) (z)
    = L_h u (z) - L_h u_h(z)
    = L_h u(z) - f(z) \\
    &= L_h u(z) - L u(z)
    = (L_h u - L u)(z)
    = r_h(z) ,
\end{align*}
\end{bem}
d.h. der Fehler erfüllt die Gleichung
\begin{align*}
L_h \eta_h(z) &= r_h \text{ in } \Omega_h \\
\eta_h &= 0 \text{ auf } \partial \Omega_h
\end{align*}

In Matrix-Schreibweise:
\[
A_h E_h = R_h
\]

Es gilt:
\[
\Vert E_h \Vert_\infty \leq \Vert A_h^{-1} \Vert_\infty \cdot \Vert R_h \Vert_\infty
\]

\begin{bem}
Hat $L_h$ die Konsistenzordnung $m$, so gilt:
\[
\Vert R_h \Vert_\infty = \mathcal O(h^m), \; h \to 0
\]
Damit $\Vert E_h \Vert_\infty \to 0$ für $h \to 0$ gilt, brauchen wir:
\begin{itemize}
\item \textbf{Konsistenz:} $\Vert R_h \Vert_\infty \to 0$ für $h \to 0$
\item \textbf{Stabilität:} $\Vert A_h^{-1} \Vert_\infty \leq C$ für $h \to 0$, wobei die Konstante
$C$ unabhängig von $h$ ist.
\end{itemize}
\[
\text{Konsistenz } + \text{ Stabilität } \implies \text{ Konvergenz}
\]
\end{bem}

\begin{lem}
Sei $\Omega = (0, 1) \times (0, 1)$ und $R > 0$ so, dass $\Omega \subset \mathbb B(0, R) = 
\{ (x,y) \in \mathbb R^2 : \ x^2 + y^2 \leq R^2 \}$.
Sei $v_h: \overline \Omega_h \to \mathbb R$ die Lösung von
\begin{align*}
- \Delta_h v_h &= 1 \text{ in } \Omega_h \\
v_h &= 0 \text{ auf } \partial \Omega_h
\end{align*}
Dann gilt für alle $z = (x, y) \in \overline \Omega_h$:
\[
0 \leq v_h(z) \leq \frac 1 4 \left( R^2 - \Vert z \Vert^2 \right)
\]
\end{lem}
\begin{proof}
Seien 
\begin{align*}
w(x, y) &:= \frac{ 1}{4} \left( R^2 - (x^2 + y^2) \right) \text{ für } (x,y) \in \overline \Omega \\
w_h(x_i, y_j) &:= w(x_i, y_j) \text{ für } (x_i, y_j) \in \overline \Omega_h .
\end{align*}
Es ist
\begin{align*}
- \Delta w(x,y) &= - \left( \frac{ \partial^2}{\partial x^2} + \frac{\partial^2}{\partial y^2} \right)
    \cdot \frac 1 4 \left( R^2 - (x^2 + y^2) \right) \\
    &= \frac{\partial }{\partial x} \left( \frac 1 2 x \right) + \frac{\partial }{\partial y}
    \left( \frac 1 2 y \right)
    = \frac 1 2 + \frac 1 2 = 1
\end{align*}
Unter Verwendung des Fünf-Punkte-Sterns, rechnet man nach, dass auch
\[
- \Delta_h w_h = 1
\]
gilt.
Da $w_h | \partial \Omega_h = \frac 1 4 \left( R^2 - (x^2 + y^2) \right) \geq 0$, erhalten wir
\begin{align*}
- \Delta_h (v_h - w_h) &= 0 \text{ in } \Omega_h \\
v_h - w_h &\leq 0 \text{ auf } \partial \Omega_h
\end{align*}
Das Maximumsprinzip liefert nun
\[
v_h(z) - w_h(z) \leq 0 \text{ für } z \in \partial \Omega_h .
\]
Ausserdem gilt
\begin{align*}
- \Delta_h (-v_h) &= -1 \text{ in } \Omega_h \\
-v_h &= 0 \text{ auf } \partial \Omega_h
\end{align*}
Also ist nach dem Maximumsprinzip
\[
v_h(z) \leq 0 \text{ für } z \in \Omega_h
\]
Insgesamt haben wir also
\[
0 \leq v_h(z) \leq w_h(z) \text{ für } z \in \overline \Omega_h
\]
\end{proof}

\begin{satz}[Konvergenz]
Seien $\Omega = (0,1) \times (0,1)$ und $u: \Omega \to \mathbb R$, $u_h: \Omega_h \to \mathbb R$
Lösungen von
\begin{align*}
- \Delta u &= f \text{ in } \Omega \\
u &= g \text{ auf } \partial \Omega
\end{align*}
und
\begin{align*}
- \Delta_h u_h &= f \text{ in } \Omega_h \\
u_h &= g \text{ auf } \partial \Omega_h
\end{align*}

Dann gilt:
\begin{align*}
\Vert u - u_h \Vert_\infty &\leq Ch^2, \; h \to 0, \; u \in C^4( \overline \Omega) \\
\Vert u - u_h \Vert_\infty &\leq Ch, \; h \to 0, \; u \in C^3( \overline \Omega) \\
\Vert u - u_h \Vert_\infty &\to 0, \; h \to 0, \; u \in C^2( \overline \Omega)
\end{align*}
\end{satz}

\begin{proof}
Sei $\eta_h(z) := u(z) - u_h(z)$, $z \in \overline \Omega_h$ der Fehler.

Wir wissen:
\begin{align*}
- \Delta_h \eta_h &= r_h := \Delta u - \Delta_h u \text{ in } \Omega_h \\
\eta_h &= 0 \text{ auf } \partial \Omega_h
\end{align*}

Weiter gilt (siehe Serie 3):
\begin{align*}
\Vert r_h \Vert_\infty &\leq Ch^2, \; h \to 0, \; u \in C^4( \overline \Omega) \\
\Vert r_h \Vert_\infty &\leq Ch, \; h \to 0, \; u \in C^3( \overline \Omega) \\
\Vert r_h \Vert_\infty &\to 0, \; h \to 0, \; u \in C^2( \overline \Omega) \\
\end{align*}

Wir müssen die Stabilität beweisen, d.h. $\Vert \eta_h \Vert_\infty \leq C \Vert r_h \Vert_\infty$.
Dazu betrachten wir
\begin{align*}
- \Delta_h v_h &= 1 \text{ in } \Omega_h \\
v_h &= 0 \text{ auf } \partial \Omega_h 
\end{align*}

Es folgt:
\begin{align*}
- \Delta_h \left( \pm \eta_h - \Vert r_h \Vert_\infty v_h \right) = \left( \pm r_h - \Vert r_h \Vert_\infty
 \right) &\leq 0 \text{ in } \Omega_h \\
 \pm \eta_h - \Vert r_h \Vert_\infty v_h &= 0 \text{ auf } \partial \Omega_h
\end{align*}

Das Maximumsprinzip impliziert nun
\[
\pm \eta_h - \Vert r_h \Vert_\infty v_h \leq 0 \text{ in } \overline \Omega_h .
\]
Also ist auch 
\[
| \eta_h (z) | \leq \Vert r_h \Vert_\infty | v_h (z) |
\]
für $z \in \overline \Omega_h$.

Nun folgt mit dem Lemma:
\[
\Vert \eta_h \Vert_\infty \leq \Vert r_h \Vert_\infty \Vert v_h \Vert_\infty
    \leq \Vert r_h \Vert_\infty \cdot \frac 1 4 R^2
\]
\end{proof}

\begin{bem}
\begin{enumerate}
\item Der obige Beweis gilt auch für allgemeine Gebiete, die diskret zusammenhängend sind. Der Beweis
des Maximumsprinzip und des Lemmas sind identisch.
\item Finite Differenzen Verfahren sind extrem effizient auf regelmässigen Gittern und einfachen 
Geometrien. Sie lassen sich allerdings nur schwer anwenden bei
\begin{itemize}
\item Krummen Rändern
\item Unstrukturierten Gittern
\item Lokalen Verfeinerungen
\end{itemize}
\item Verahren höherer Ordnung benutzen mehr Punkte, z.B. der Neun-Punkte-Stern. Dies wiederum führt zu
Problemen bei Randbedingungen.
\end{enumerate}
\end{bem}







\chapter{Finite Elemente Verfahren für die Poisson-Gleichung}
\section{Variationsformulierung in 1D}

Betrachte
\begin{align*}
-u'' &= f \text{ in } \Omega = (0,1) \\
u(0) &= 0 \\ 
u'(1) &= 0
\end{align*}

Durch Integration erhalten wir:
\begin{align*}
u'(\xi) &= - \int_0^\xi f(t) \ \mathrm dt + u'(0) \\
u(x) &= - \int_0^x \int_0^\xi f(t) \ \mathrm dt \ \mathrm d\xi + x u'(0) + \underbrace{u(0)}_{=0} \\
0 &= u'(1) = - \int_0^1 f(t) \ \mathrm dt = u'(0)
\end{align*}

Wir erhalten also die Lösungsformel
\[
u(x) = x \int_0^1 f(t) \ \mathrm dt - \int_0^x \int_0^\xi f(t) \ \mathrm dt \ \mathrm d\xi \label{eq:lsg}
\]

\begin{bem}
Wir sehen: Falls $f \in C^k(\overline \Omega)$, dann ist $u \in C^{k+2} (\overline \Omega)$.
\end{bem}

\begin{defi}
$f: \Omega \to \mathbb R$ heisst \textbf{stückweise} $C^k$, wenn es endlich viele offene 
Teilgebiete $\Omega_i \subset \Omega$ gibt mit $f_{|\Omega_i} \in C^k(\overline \Omega_i)$.
Wir schreiben $f \in C^k_{stw}(\Omega)$.
\end{defi}

\begin{satz}
Sei $u$ gegeben durch (\ref{eq:lsg}) mit $f \in C^0_{stw}(\Omega)$. Dann ist
$u \in C^2_{stw} \cap C^1(\Omega)$.
\end{satz}
\begin{proof}
Klar.
\end{proof}

\begin{bem}
Die Lösungsformel (\ref{eq:lsg}) hat eine Sinn auf $\overline \Omega$, auch wenn die 
Differentialgleichung bei $x = s_i$ nicht gilt.
\end{bem}

\begin{idee}
Verlange (\ref{eq:lsg}) nicht punktweise sondern im Mittel. Dazu multipliziere 
$f$ mit einer Testfunktion $v \in C^1(\Omega)$, $v(0) = 0$ und integriere
partiell:
\[
\int_0^1 f(x) v(x) \ \mathrm dx = - \int_0^1 v(x) u''(x) \ \mathrm dx
    = \int_0^1 u'(x) v'(x) \ \mathrm dx - \underbrace{ \left[v(x) u'(x) \right]_0^1}_{=0}
\]
\end{idee}

Seien
\[
a(u,v) := \int_0^1 u'(x) v'(x) \ \mathrm dx
\]
und
\[
\ell(v) := \int_0^1 f(x) v(x) \ \mathrm dx \ .
\]

\textbf{Variationsformulierung:} Finde $u \in V$ ($V$ ist noch nicht näher definiert), so dass
\[
    a(u,v) = \ell(v), \; v \in V \label{eq:var} \tag{V}
\]

Für $u, v, w \in C^1(\overline \Omega)$, $\lambda \in \mathbb R$ gelten:
\begin{align*}
a(u+v, w) &= a(u,w) + a(v,w) \\
a(u, v+w) &= a(u,v) + a(u,w) \\
a(\lambda u, v) &= \lambda a(u,v) = a(u, \lambda v)
\end{align*}
Also ist $w$ eine Bilinearform.

\[
\ell(v + \lambda w) = \ell(v) + \lambda l(w)
\]
Also ist $\ell$ ein lineares Funktional.

Offen ist noch die Wahl von $V$.

\subsection{Sobolevräume in 1D}
\begin{idee}
Lineare Funktionalräume mit \textit{endlicher Energie}.
\[
H^k(\Omega) \supsetneqq C^k(\overline \Omega)
\]
\end{idee}

\begin{defi}
Sei $u \in C^0(\overline \Omega)$. Dann heisst $v: \Omega \to \mathbb R$ \textbf{schwache Ableitung} 
von $u$, falls
\[
\int_\Omega v \varphi \ \mathrm dx = - \int_\Omega u \varphi' \ \mathrm dx, \; \varphi \in C_0^1(\overline \Omega) \ .
\]
Analog heisst $v$ $k$-te schwache Ableitung von $u$, falls
\[
\int_\Omega v \varphi \ \mathrm dx = (-1)^k \int_\Omega u \varphi^{(k)} \ \mathrm dx, 
    \; \varphi \in C_0^k(\overline \Omega) \ ,
\]
wobei
\[
C_0^k(\overline \Omega) = \{ \varphi \in C^k(\overline \Omega) \ : \ 
    \varphi^{(j)}(0) = \varphi^{(j)}(1) = 0, \ 0 \leq j \leq k-1 \}
\]
\end{defi}

\begin{bspe}
    \begin{enumerate}
        \item Sei $u(x) := |x|$, $\Omega = (-1,1)$, dann ist
            \[
                v(x) = \begin{cases}
                    -1, & x < 0 \\
                    \text{nicht definiert}, & x = 0 \\
                    1, & x > 0
                \end{cases}
            \]
            die schwache Ableitung von $u$.
            Denn für $\varphi \in C_0^1(\overline \Omega)$ ist
            \[
                \int_{-1}^1 v(x) \varphi(x) \ \mathrm dx
                    = \int_{-1}^0 v(x) \varphi(x) \ \mathrm dx + \int_0^1 v(x) \varphi(x) \ \mathrm dx
                    = - \int_{-1}^0 \varphi(x) \ \mathrm dx + \int_0^1 \varphi(x) \ \mathrm dx
            \]
            und
            \begin{align*}
                - \int_{-1}^1 u(x) \varphi'(x) \ \mathrm dx
                &= \int_{-1}^0 x \varphi'(x) \mathrm dx - \int_0^1 x \varphi'(x) \ \mathrm dx \\
                &= - \int_{-1}^0 \varphi(x) \ \mathrm dx + \left[ x \varphi(x) \right]_{-1}^0
                + \int_0^1 \varphi(x) \ \mathrm dx - \left[ x \varphi(x) \right]_0^1 \ .
            \end{align*}
        \item Sei $u \in C^0(\overline \Omega) \cap C_{stw}^1(\Omega)$. Dann hat $u$ eine schwache Ableitung $u'$ gegeben
            durch
            \[
                u'(x) = \begin{cases}
                    \frac{\mathrm du}{\mathrm dx}(x), & x_{i-1} < x < s_i \\
                    \text{nicht definiert}, & x = s_i
                \end{cases}
            \]
        \item Falls $u \in C^1(\overline \Omega)$, d.h. $u$ ist differenzierbar im üblichen Sinn, dann ist die schwache
            Ableitung gleich der klassischen Ableitung.
    \end{enumerate}
\end{bspe}

\begin{konv}
Alle Ableitungen werden als schwache Ableitungen verstanden.
\end{konv}

\begin{defi}
    Sei $\Omega = (0,1)$. Dann ist
    \begin{align*}
        H^1(\Omega) &:= \{ u \in L^2(\Omega) \ : \ u' \in L^2(\Omega) \} \\
        H^k(\Omega) &:= \{ u \in L^2(\Omega) \ : \ u^{(j)} \in L^2(\Omega), \ 0 \leq j \leq k \} \\
        H^0(\Omega) &:= L^2(\Omega) := \left\{ u: \Omega \to \mathbb R \text{ messbar } : \ \int_\Omega |u|^2 \ \mathrm dx < \infty \right\} \\
        \Vert u \Vert_{L^2(\Omega)}^2 &= \int_\Omega | u(x) |^2 \ \mathrm dx
    \end{align*}
\end{defi}

\begin{bem}
    Es gilt:
    \[
        L^2(\Omega) \supset H^1(\Omega) \supset H^2(\Omega) \supset \ldots \supset H^k(\Omega) \supset \ldots
    \]
\end{bem}

\begin{satz}
    $L^2(\Omega)$ ist ein Hilbertraum für das Skalarprodukt
    \[
        (u, v) := \int_\Omega u(x) v(x) \ \mathrm dx \ ,
    \]
    d.h. ein vollständiger linearer normierter Vektorraum (Banachraum) bezüglich der Norm
    \[
        \Vert u \Vert_{L^2(\Omega)} := \sqrt{ (u, u) } \ .
    \]
    Es gilt die Cauchy-Schwarzsche Ungleichung:
    \[
        | (u,v) | \leq \Vert u \Vert_{L^2(\Omega)} \Vert v \Vert_{L^2(\Omega)}
    \]
\end{satz}
\begin{proof}
Siehe Funktionalanalysis, Übungen.
\end{proof}

\begin{bem}
    \begin{enumerate}
        \item Man identifiziert $u, v \in L^2(\Omega)$ falls $u(x) = v(x)$ fast überall in $\Omega$, d.h.
            $\Vert u - v \Vert_{L^2(\Omega)} = 0$.
            Falls $u \in L^2(\Omega)$, macht es im Allgemeinen keinen Sinn von Punktwerten von $u(x)$ zu sprechen.
            Insbesondere sind $u(0)$ oder $u(1)$ nicht wohldefiniert.
        \item Sei $u \in L^2(\Omega)$.
            \[
                \int_\Omega u \varphi = 0, \; \forall \varphi \in C^\infty(\Omega) \quad \implies \quad u = 0 \text{ (im $L^2$-Sinn)}
            \]
    \end{enumerate}
\end{bem}

\begin{satz}
    Für $k \geq 0$ ist $H^k(\Omega)$ ein Hilbertraum für das Skalarprodukt
    \[
        (u, v) := \sum_{j=0}^k ( u^{(j)}, v^{(j)} )_{L^2(\Omega)}
    \]
    Die induzierte Norm ist gegeben durch
    \[
        \Vert u \Vert_{H^k(\Omega)} := (u, u)^{\frac 1 2} = \left( \sum_{j=0}^k \Vert u^{(j)} \Vert_{L^2(\Omega)}^2 \right)^{\frac 1 2}
    \]
    Weiter ist die $H^k(\Omega)$-Seminorm gegeben durch
    \[
        \vert u \vert_{H^k(\Omega)} := \Vert u^{(k)} \Vert_{L^2(\Omega)}
    \]
\end{satz}

\begin{bspe}
    \begin{enumerate}
        \item Wenn $u \in H^1(\Omega)$, dann ist
            \begin{align*}
                \Vert u \Vert_{H^1(\Omega)} &= \Vert u \Vert_1 = \sqrt{ \Vert u \Vert_{L^2}^2 + \Vert u' \Vert_{L^2}^2 } \\
                \vert u \vert_{H^1(\Omega)} &= \vert u \vert_1 = \Vert u' \Vert_{L^2(\Omega)}
            \end{align*}
            Wenn $\Omega = (0,1)$ und $u = 1$ auf $\Omega$, dann ist $|u|_1 = 0$ aber $\Vert u \Vert_{H^1(\Omega)} = 1$.
        \item $u(x) = |x - \frac 1 2 |^{- \frac 1 4} \in L^2(\Omega)$, aber $u(\frac 1 2) = \infty$.
    \end{enumerate}
\end{bspe}

\begin{satz}
    Sei $u \in H^1(\Omega)$, $\Omega = (0,1)$. Dann existiert ein $\tilde u \in C(\overline \Omega)$ mit $u(x) = \tilde u(x)$
    fast überall.
\end{satz}

\begin{bem}
    \begin{enumerate}
        \item Falls $u \in H^1(\Omega)$, macht es Sinn von Punkt- und Randwerten von $u$ zu sprechen. Damit sind immer die 
            Werte von $\tilde u$ gemeint ($u = \tilde u$ in $L^2$).
        \item $H^1(\Omega) \subset C^0(\overline \Omega)$
    \end{enumerate}
\end{bem}

\begin{bsp}
    Sei $\Omega = (-1, 1)$ und 
    \[
        u(x) := \begin{cases}
            0, & \text{falls } x < 0 \\
            1, & \text{falls } x > 0
        \end{cases}
    \]
    Dann ist $u \notin H^1(\Omega)$.

    Es gilt $u \in L^2(\Omega)$. Angenommen $u' \in L^2(\Omega)$, wobei $u'$ die schwache Ableitung von $u$ bezeichne.
    Dann gilt per Definition der schwachen Ableigung für jedes $\varphi \in C_0^1(\Omega)$:
    \[
        \int_{-1}^1 u' \varphi \ \mathrm  dx = - \int_{-1}^1 u \varphi' \ \mathrm dx = - \int_0^1 \varphi' \ \mathrm dx = \varphi(0)
    \]
    Aus der Cauchy-Schwarz-Ungleichung folgt:
    \[
        |\varphi(0)| = | (u', \varphi) | \leq \Vert u' \Vert_{L^2(\Omega)}  \Vert \varphi \Vert_{L^2(\Omega)} 
            \leq C \Vert \varphi \Vert_{L^2(\Omega)}
    \]
    Für jedes $C > 0$ gibt es aber ein $\varphi \in C_0^1(\Omega)$ mit
    \[
        | \varphi(0) | \geq C \Vert \varphi \Vert_{L^2(\Omega)} \ ,
    \]
    was ein Widerspruch darstellt.
\end{bsp}

\begin{satz}
    Für $m \geq 0$ gilt:
    $C^\infty(\Omega) \cap H^m(\Omega)$ liegt dicht in $H^m(\Omega)$.
\end{satz}

\begin{bem}
    Funktionen in $H^m(\Omega)$ lassen sich beliebig genau durch glatte Funktionen approximieren (bzgl. der 
    $\Vert \cdot \Vert_{H^m}$-Norm).
\end{bem}

\begin{defi}
    Die Vervollständigung von $C_0^\infty(\Omega)$ bzgl. der $\Vert \cdot \Vert_{H^m}$-Norm heisst
    $H_0^m(\Omega)$.
    \[
        C_0^\infty(\Omega) := \{ u \in C^\infty(\Omega) \ : \ \supp\{u\} \subset \subset \Omega \}
    \]
\end{defi}

\begin{bsp}
    Sei $\Omega = (0,1)$,  $H_0^1(\Omega) = \{ v \in H^1(\Omega) \ : \ v(0) = v(1) = 0 \}.$
    Sei $u(x) = x$ auf $\Omega$. Dann ist $u \in H^1(\Omega)$, aber $u \notin H_0^1(\Omega)$.

    Angenommen, $u \in H_0^1(\Omega)$. 
    Definiere $L: H^1(\Omega) \to \mathbb R$, $v \mapsto \int_0^1 v'(x) \ \mathrm dx$.
    
    Mit der Cauchy-Schwarz-Ungleichung folgt:
    \[
        |L(v)| = \left| \int_\Omega 1 \cdot v'(x) \ \mathrm dx \right|
            \leq \Vert v' \Vert_{L^2(\Omega)} \Vert 1 \Vert_{L^2(\Omega)}
            \leq \Vert v \Vert_H^1(\Omega)
    \]
    Also ist $L$ stetig.

    Sei $\{ \varphi_n \}_{n\geq 1} \subset C_0^\infty$ so, dass 
    $\Vert \varphi_n - u \Vert_{H^1(\Omega)} \to 0$ für $n \to \infty$.

    Es folgt
    \[ 
        |L(u) - L(\varphi_n) | = | L(u - \varphi_n) | \leq \Vert u - \varphi_n \Vert_{H^1(\Omega)} \to 0 \ ,
    \]
    d.h. $L(\varphi_n) \to L(u)$ für $n \to \infty$.
    Aber es gilt
    \[
        L(\varphi_n) = \int_0^1 \varphi'_n(x) \ \mathrm dx = 0 \text{ für } n \in \mathbb N \quad \text{und} \quad
        L(u) = \int_0^1 1 \ \mathrm dx = 1 \ ,
    \]
    also haben wir einen Widerspruch gefunden und damit ist $u \notin H_0^1(\Omega)$.
\end{bsp}

Wir können nun den Raum $V$ für \eqref{eq:var} festlegen:

Finde $u \in V := \{u \in H^1(\Omega) \ : \ u(0) = 0 \}$ so, dass
\[
    a(u,v) = \int_\Omega u' v' = \ell(v) = \int_\Omega f v \quad \forall v \in V
\]








\end{document}
