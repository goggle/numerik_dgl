\chapter{Finite Elemente Verfahren für die Poisson-Gleichung}
\section{Variationsformulierung in 1D}

Betrachte
\begin{align*}
-u'' &= f \text{ in } \Omega = (0,1) \\
u(0) &= 0 \\ 
u'(1) &= 0
\end{align*}

Durch Integration erhalten wir:
\begin{align*}
u'(\xi) &= - \int_0^\xi f(t) \ \mathrm dt + u'(0) \\
u(x) &= - \int_0^x \int_0^\xi f(t) \ \mathrm dt \ \mathrm d\xi + x u'(0) + \underbrace{u(0)}_{=0} \\
0 &= u'(1) = - \int_0^1 f(t) \ \mathrm dt + u'(0)
\end{align*}

Wir erhalten also die Lösungsformel
\[
u(x) = x \int_0^1 f(t) \ \mathrm dt - \int_0^x \int_0^\xi f(t) \ \mathrm dt \ \mathrm d\xi \label{eq:lsg}
\]

\begin{bem}
Wir sehen: Falls $f \in C^k(\overline \Omega)$, dann ist $u \in C^{k+2} (\overline \Omega)$.
\end{bem}

\begin{defi}
$f: \Omega \to \mathbb R$ heisst \textbf{stückweise} $C^k$, wenn es endlich viele offene 
Teilgebiete $\Omega_i \subset \Omega$ gibt mit $f_{|\Omega_i} \in C^k(\overline \Omega_i)$.
Wir schreiben $f \in C^k_{stw}(\Omega)$.
\end{defi}

\begin{satz}
Sei $u$ gegeben durch (\ref{eq:lsg}) mit $f \in C^0_{stw}(\Omega)$. Dann ist
$u \in C^2_{stw} \cap C^1(\Omega)$.
\end{satz}
\begin{proof}
Klar.
\end{proof}

\begin{bem}
Die Lösungsformel (\ref{eq:lsg}) hat eine Sinn auf $\overline \Omega$, auch wenn die 
Differentialgleichung bei $x = s_i$ nicht gilt.
\end{bem}

\begin{idee}
Verlange (\ref{eq:lsg}) nicht punktweise sondern im Mittel. Dazu multipliziere 
$f$ mit einer Testfunktion $v \in C^1(\Omega)$, $v(0) = 0$ und integriere
partiell:
\[
\int_0^1 f(x) v(x) \ \mathrm dx = - \int_0^1 v(x) u''(x) \ \mathrm dx
    = \int_0^1 u'(x) v'(x) \ \mathrm dx - \underbrace{ \left[v(x) u'(x) \right]_0^1}_{=0}
\]
\end{idee}

Seien
\[
a(u,v) := \int_0^1 u'(x) v'(x) \ \mathrm dx
\]
und
\[
\ell(v) := \int_0^1 f(x) v(x) \ \mathrm dx \ .
\]

\textbf{Variationsformulierung:} Finde $u \in V$ ($V$ ist noch nicht näher definiert), so dass
\[
    a(u,v) = \ell(v), \; v \in V \label{eq:var} \tag{V}
\]

Für $u, v, w \in C^1(\overline \Omega)$, $\lambda \in \mathbb R$ gelten:
\begin{align*}
a(u+v, w) &= a(u,w) + a(v,w) \\
a(u, v+w) &= a(u,v) + a(u,w) \\
a(\lambda u, v) &= \lambda a(u,v) = a(u, \lambda v)
\end{align*}
Also ist $w$ eine Bilinearform.

\[
\ell(v + \lambda w) = \ell(v) + \lambda l(w)
\]
Also ist $\ell$ ein lineares Funktional.

Offen ist noch die Wahl von $V$.

\subsection{Sobolevräume in 1D}
\begin{idee}
Lineare Funktionalräume mit \textit{endlicher Energie}.
\[
H^k(\Omega) \supsetneqq C^k(\overline \Omega)
\]
\end{idee}

\begin{defi}
Sei $u \in C^0(\overline \Omega)$. Dann heisst $v: \Omega \to \mathbb R$ \textbf{schwache Ableitung} 
von $u$, falls
\[
\int_\Omega v \varphi \ \mathrm dx = - \int_\Omega u \varphi' \ \mathrm dx, \; \varphi \in C_0^1(\overline \Omega) \ .
\]
Analog heisst $v$ $k$-te schwache Ableitung von $u$, falls
\[
\int_\Omega v \varphi \ \mathrm dx = (-1)^k \int_\Omega u \varphi^{(k)} \ \mathrm dx, 
    \; \varphi \in C_0^k(\overline \Omega) \ ,
\]
wobei
\[
C_0^k(\overline \Omega) = \{ \varphi \in C^k(\overline \Omega) \ : \ 
    \varphi^{(j)}(0) = \varphi^{(j)}(1) = 0, \ 0 \leq j \leq k-1 \}
\]
\end{defi}

\begin{bspe}
    \begin{enumerate}
        \item Sei $u(x) := |x|$, $\Omega = (-1,1)$, dann ist
            \[
                v(x) = \begin{cases}
                    -1, & x < 0 \\
                    \text{nicht definiert}, & x = 0 \\
                    1, & x > 0
                \end{cases}
            \]
            die schwache Ableitung von $u$.
            Denn für $\varphi \in C_0^1(\overline \Omega)$ ist
            \[
                \int_{-1}^1 v(x) \varphi(x) \ \mathrm dx
                    = \int_{-1}^0 v(x) \varphi(x) \ \mathrm dx + \int_0^1 v(x) \varphi(x) \ \mathrm dx
                    = - \int_{-1}^0 \varphi(x) \ \mathrm dx + \int_0^1 \varphi(x) \ \mathrm dx
            \]
            und
            \begin{align*}
                - \int_{-1}^1 u(x) \varphi'(x) \ \mathrm dx
                &= \int_{-1}^0 x \varphi'(x) \mathrm dx - \int_0^1 x \varphi'(x) \ \mathrm dx \\
                &= - \int_{-1}^0 \varphi(x) \ \mathrm dx + \left[ x \varphi(x) \right]_{-1}^0
                + \int_0^1 \varphi(x) \ \mathrm dx - \left[ x \varphi(x) \right]_0^1 \ .
            \end{align*}
        \item Sei $u \in C^0(\overline \Omega) \cap C_{stw}^1(\Omega)$. Dann hat $u$ eine schwache Ableitung $u'$ gegeben
            durch
            \[
                u'(x) = \begin{cases}
                    \frac{\mathrm du}{\mathrm dx}(x), & x_{i-1} < x < s_i \\
                    \text{nicht definiert}, & x = s_i
                \end{cases}
            \]
        \item Falls $u \in C^1(\overline \Omega)$, d.h. $u$ ist differenzierbar im üblichen Sinn, dann ist die schwache
            Ableitung gleich der klassischen Ableitung.
    \end{enumerate}
\end{bspe}

\begin{konv}
Alle Ableitungen werden als schwache Ableitungen verstanden.
\end{konv}

\begin{defi}
    Sei $\Omega = (0,1)$. Dann ist
    \begin{align*}
        H^1(\Omega) &:= \{ u \in L^2(\Omega) \ : \ u' \in L^2(\Omega) \} \\
        H^k(\Omega) &:= \{ u \in L^2(\Omega) \ : \ u^{(j)} \in L^2(\Omega), \ 0 \leq j \leq k \} \\
        H^0(\Omega) &:= L^2(\Omega) := \left\{ u: \Omega \to \mathbb R \text{ messbar } : \ \int_\Omega |u|^2 \ \mathrm dx < \infty \right\} \\
        \Vert u \Vert_{L^2(\Omega)}^2 &= \int_\Omega | u(x) |^2 \ \mathrm dx
    \end{align*}
\end{defi}

\begin{bem}
    Es gilt:
    \[
        L^2(\Omega) \supset H^1(\Omega) \supset H^2(\Omega) \supset \ldots \supset H^k(\Omega) \supset \ldots
    \]
\end{bem}

\begin{satz}
    $L^2(\Omega)$ ist ein Hilbertraum für das Skalarprodukt
    \[
        (u, v) := \int_\Omega u(x) v(x) \ \mathrm dx \ ,
    \]
    d.h. ein vollständiger linearer normierter Vektorraum (Banachraum) bezüglich der Norm
    \[
        \Vert u \Vert_{L^2(\Omega)} := \sqrt{ (u, u) } \ .
    \]
    Es gilt die Cauchy-Schwarzsche Ungleichung:
    \[
        | (u,v) | \leq \Vert u \Vert_{L^2(\Omega)} \Vert v \Vert_{L^2(\Omega)}
    \]
\end{satz}
\begin{proof}
Siehe Funktionalanalysis, Übungen.
\end{proof}

\begin{bem}
    \begin{enumerate}
        \item Man identifiziert $u, v \in L^2(\Omega)$ falls $u(x) = v(x)$ fast überall in $\Omega$, d.h.
            $\Vert u - v \Vert_{L^2(\Omega)} = 0$.
            Falls $u \in L^2(\Omega)$, macht es im Allgemeinen keinen Sinn von Punktwerten von $u(x)$ zu sprechen.
            Insbesondere sind $u(0)$ oder $u(1)$ nicht wohldefiniert.
        \item Sei $u \in L^2(\Omega)$.
            \[
                \int_\Omega u \varphi = 0, \; \forall \varphi \in C^\infty(\Omega) \quad \implies \quad u = 0 \text{ (im $L^2$-Sinn)}
            \]
    \end{enumerate}
\end{bem}

\begin{satz}
    Für $k \geq 0$ ist $H^k(\Omega)$ ein Hilbertraum für das Skalarprodukt
    \[
        (u, v) := \sum_{j=0}^k ( u^{(j)}, v^{(j)} )_{L^2(\Omega)}
    \]
    Die induzierte Norm ist gegeben durch
    \[
        \Vert u \Vert_{H^k(\Omega)} := (u, u)^{\frac 1 2} = \left( \sum_{j=0}^k \Vert u^{(j)} \Vert_{L^2(\Omega)}^2 \right)^{\frac 1 2}
    \]
    Weiter ist die $H^k(\Omega)$-Seminorm gegeben durch
    \[
        \vert u \vert_{H^k(\Omega)} := \Vert u^{(k)} \Vert_{L^2(\Omega)}
    \]
\end{satz}

\begin{bspe}
    \begin{enumerate}
        \item Wenn $u \in H^1(\Omega)$, dann ist
            \begin{align*}
                \Vert u \Vert_{H^1(\Omega)} &= \Vert u \Vert_1 = \sqrt{ \Vert u \Vert_{L^2}^2 + \Vert u' \Vert_{L^2}^2 } \\
                \vert u \vert_{H^1(\Omega)} &= \vert u \vert_1 = \Vert u' \Vert_{L^2(\Omega)}
            \end{align*}
            Wenn $\Omega = (0,1)$ und $u = 1$ auf $\Omega$, dann ist $|u|_1 = 0$ aber $\Vert u \Vert_{H^1(\Omega)} = 1$.
        \item $u(x) = |x - \frac 1 2 |^{- \frac 1 4} \in L^2(\Omega)$, aber $u(\frac 1 2) = \infty$.
    \end{enumerate}
\end{bspe}

\begin{satz}
    Sei $u \in H^1(\Omega)$, $\Omega = (0,1)$. Dann existiert ein $\tilde u \in C(\overline \Omega)$ mit $u(x) = \tilde u(x)$
    fast überall.
\end{satz}

\begin{bem}
    \begin{enumerate}
        \item Falls $u \in H^1(\Omega)$, macht es Sinn von Punkt- und Randwerten von $u$ zu sprechen. Damit sind immer die 
            Werte von $\tilde u$ gemeint ($u = \tilde u$ in $L^2$).
        \item $H^1(\Omega) \subset C^0(\overline \Omega)$
    \end{enumerate}
\end{bem}

\begin{bsp}
    Sei $\Omega = (-1, 1)$ und 
    \[
        u(x) := \begin{cases}
            0, & \text{falls } x < 0 \\
            1, & \text{falls } x > 0
        \end{cases}
    \]
    Dann ist $u \notin H^1(\Omega)$.

    Es gilt $u \in L^2(\Omega)$. Angenommen $u' \in L^2(\Omega)$, wobei $u'$ die schwache Ableitung von $u$ bezeichne.
    Dann gilt per Definition der schwachen Ableigung für jedes $\varphi \in C_0^1(\Omega)$:
    \[
        \int_{-1}^1 u' \varphi \ \mathrm  dx = - \int_{-1}^1 u \varphi' \ \mathrm dx = - \int_0^1 \varphi' \ \mathrm dx = \varphi(0)
    \]
    Aus der Cauchy-Schwarz-Ungleichung folgt:
    \[
        |\varphi(0)| = | (u', \varphi) | \leq \Vert u' \Vert_{L^2(\Omega)}  \Vert \varphi \Vert_{L^2(\Omega)} 
            \leq C \Vert \varphi \Vert_{L^2(\Omega)}
    \]
    Für jedes $C > 0$ gibt es aber ein $\varphi \in C_0^1(\Omega)$ mit
    \[
        | \varphi(0) | \geq C \Vert \varphi \Vert_{L^2(\Omega)} \ ,
    \]
    was ein Widerspruch darstellt.
\end{bsp}

\begin{satz}
    Für $m \geq 0$ gilt:
    $C^\infty(\Omega) \cap H^m(\Omega)$ liegt dicht in $H^m(\Omega)$.
\end{satz}

\begin{bem}
    Funktionen in $H^m(\Omega)$ lassen sich beliebig genau durch glatte Funktionen approximieren (bzgl. der 
    $\Vert \cdot \Vert_{H^m}$-Norm).
\end{bem}

\begin{defi}
    Die Vervollständigung von $C_0^\infty(\Omega)$ bzgl. der $\Vert \cdot \Vert_{H^m}$-Norm heisst
    $H_0^m(\Omega)$.
    \[
        C_0^\infty(\Omega) := \{ u \in C^\infty(\Omega) \ : \ \supp\{u\} \subset \subset \Omega \}
    \]
\end{defi}

\begin{bsp}
    Sei $\Omega = (0,1)$,  $H_0^1(\Omega) = \{ v \in H^1(\Omega) \ : \ v(0) = v(1) = 0 \}.$
    Sei $u(x) = x$ auf $\Omega$. Dann ist $u \in H^1(\Omega)$, aber $u \notin H_0^1(\Omega)$.

    Angenommen, $u \in H_0^1(\Omega)$. 
    Definiere $L: H^1(\Omega) \to \mathbb R$, $v \mapsto \int_0^1 v'(x) \ \mathrm dx$.
    
    Mit der Cauchy-Schwarz-Ungleichung folgt:
    \[
        |L(v)| = \left| \int_\Omega 1 \cdot v'(x) \ \mathrm dx \right|
            \leq \Vert v' \Vert_{L^2(\Omega)} \Vert 1 \Vert_{L^2(\Omega)}
            \leq \Vert v \Vert_H^1(\Omega)
    \]
    Also ist $L$ stetig.

    Sei $\{ \varphi_n \}_{n\geq 1} \subset C_0^\infty$ so, dass 
    $\Vert \varphi_n - u \Vert_{H^1(\Omega)} \to 0$ für $n \to \infty$.

    Es folgt
    \[ 
        |L(u) - L(\varphi_n) | = | L(u - \varphi_n) | \leq \Vert u - \varphi_n \Vert_{H^1(\Omega)} \to 0 \ ,
    \]
    d.h. $L(\varphi_n) \to L(u)$ für $n \to \infty$.
    Aber es gilt
    \[
        L(\varphi_n) = \int_0^1 \varphi'_n(x) \ \mathrm dx = 0 \text{ für } n \in \mathbb N \quad \text{und} \quad
        L(u) = \int_0^1 1 \ \mathrm dx = 1 \ ,
    \]
    also haben wir einen Widerspruch gefunden und damit ist $u \notin H_0^1(\Omega)$.
\end{bsp}

Wir können nun den Raum $V$ für \eqref{eq:var} festlegen:

Finde $u \in V := \{u \in H^1(\Omega) \ : \ u(0) = 0 \}$ so, dass
\[
    a(u,v) = \int_\Omega u' v' = \ell(v) = \int_\Omega f v \quad \forall v \in V
\]

\begin{satz}[(V) $\implies$ (P)]
    Sei $f \in C(\overline \Omega)$ und $u \in C^2(\overline \Omega)$. Falls $u$ eine Lösung von \eqref{eq:var} ist, so ist $u$ auch eine 
    Lösung der starken From, d.h.
    \begin{align*}
        -u'' &= f \text{ in } \Omega \\
        u(0) &= 0 \\
        u'(1) &= 0
    \end{align*}
\end{satz}
\begin{proof}
    Wir wissen:
    \[
        \int_\Omega u' v' = \int_\Omega f v \quad \forall v \in V
    \]
    Sei $v \in V \cap C(\overline \Omega)$.
    \[
        \int_\Omega f v \ \mathrm dx
            = \int_\Omega u' v' \ \mathrm dx
            = - \int_0^1 u'' v \ \mathrm dx +  u'(1) v(1) - \underbrace{u'(0) v(0)}_{=0} \ ,
    \]
    d.h. für $v \in V \cap C(\overline \Omega)$ mit $v(1) = 0$ gilt:
    \[
        \int_0^1 \left( f(x) + u''(x) \right) v(x) \ \mathrm dx = 0
    \]
    Da $w(x) := f(x) + u''(x)$ auf $[0,1]$ stetig ist, folgt $w = 0$.

    Da $u \in V$, ist $u(0) = 0$ und für $v(x) := x \in V$ haben wir
    \[
        0 = \int_0^1 \left( f(x) + u''(x) \right) v(x) \ \mathrm dx = u'(1) v(1) = u'(1) \ .
    \]
    Also erfüllt $u$ automatisch die homogene Neumann-Randbedingung bei $x=1$. Deshalb heissen 
    Neumann-Randbedingungen auch natürliche Randbedingungen.
\end{proof}

\section{Die Finite Elemente Methode in 1D}

Betrachte das Modellproblem
\begin{align*}
    -u''(x) &= f(x) \text{ in } \Omega = (0,1) \\
    u(0) = u(1) &= 0
    \label{eq:p}
    \tag{P}
\end{align*}

und die Variationsformulierung:

Finde $u \in V = H_0^1(\Omega)$ so, dass 
\[
    a(u,v) = \ell(v) \label{eq:var} \tag{V}
\]
für alle $v \in V$ mit
\begin{align*}
    a(u,v) &:= \int_\Omega u'(x) v'(x) \ \mathrm dx \\
    \ell(v) &:= \int_\Omega f(x) v(x) \ \mathrm dx
\end{align*}


\subsubsection*{Galerkin Verfahren}
Wähle $V_h \subset V$ Teilraum von $V$ mit $\dim V_h < \infty$. Ersetze $V$ durch $V_h$ in \eqref{eq:var}. 
Finde $u_h \in V_h$ so, dass
\[
    a(u_h, v) = \ell(v) \quad \forall v \in V_h
\]

\textbf{Wahl von $V_h$:} Sei $K = (a, b) \subset \mathbb R$, 
\[
    \mathcal P^\ell := \{ p(x) \text{ Polynom } : \ p(x) = a_0 + a_1 x + \ldots + a_\ell x^\ell \} \ .
\]
Wir wählen
\[
    V_h := \{ v \in V \ : \ v_{|(x_i, x_{i+1})} \in \mathcal P^1( (x_i, x_{i+1}) ), \ i = 0, \ldots N \} \ ,
\]
d.h. wenn $v \in V_h$, dann ist $v(0) = v(1) = 0$, $v \in C^0(\Omega)$ ($H_0^1(\Omega) \subset C^0(\overline \Omega)$) 
und $v$ ist stückweise linear.

Sei $\{ \varphi_j \}_{j=1}^N$ eine Basis von $V_h$:
\[
    \varphi_j = 
    \begin{cases}
        1 - \frac{|x - x_j|}{h}, & x_j \leq x \leq x_{j+1} \\
        0, & \text{sonst}
    \end{cases}
\]
Es gilt $\varphi_j(x_k) = \delta_{jk}$ und jedes $v \in V_h$ lässt sich eindeutig schreiben als
\[
    v(x) = \sum_{j=1}^N v_j \varphi_j(x) \ , \quad v_j \in \mathbb R
\]
Insbesondere ist 
\[
    v(x_k) = \sum_{j=1}^N v_j \varphi_j(x_k) = v_k
\]

\textbf{Galerkin Approximation $u_h \in V_h$:}

\begin{align*}
    a(u_h, v) = \ell(v) \quad \forall v \in V_h \quad
        &\Leftrightarrow \quad a(u_h, \varphi_i) = \ell( \varphi_i) \quad \forall i = 1, \ldots, N  \\
        &\Leftrightarrow \quad a\left( \sum_{j=1}^N u_j \varphi_j, \varphi_i \right) = \ell(\varphi_i) \quad \forall i=1, \ldots, N \\
        &\Leftrightarrow \quad \sum_{j=1}^N a(\varphi_j, \varphi_i) u_j = \ell(\varphi_i) \quad \forall i=1, \ldots, N \\
        &\Leftrightarrow \quad \mat A \vec u = \vec \ell \ ,
\end{align*}
wobei
\[
    \mat A = (a_{ij})_{i,j = 1,\ldots, N} = ( a(\varphi_i, \varphi_j))_{i,j=1,\ldots,N}, \quad
    \vec u = \begin{bmatrix} u_1 \\ \vdots \\ u_N \end{bmatrix}, \quad
    \vec \ell = \begin{bmatrix} \ell(\varphi_1) \\ \vdots \\ \ell(\varphi_N) \end{bmatrix}
\]
Die $(N \times N)$-Matrix $\mat A$ heisst \textbf{Steifigkeitsmatrix} und der Vektor $\vec \ell$ heisst
\textbf{Lastvektor}.

\vspace{1em}
Bestimme $\mat A = (a_{ij})$:
\begin{align*}
    a_{ii} &= a(\varphi_i, \varphi_i) = \int_\Omega \varphi_i' \varphi_i' \ \mathrm dx
        = \int_{x_{i-1}}^{x_{i+1}} \varphi_i'(x)^2 \ \mathrm dx \\
        &= \int_{x_{i-1}}^{x_i} \varphi_i'(x)^2 \ \mathrm dx + \int_{x_i}^{x_{i+1}} \varphi_i'(x)^2 \ \mathrm dx
        = h \left(- \frac 1 h \right)^2 + h \left(\frac 1 h\right)^2 = \frac 2 h \\
    a_{ij} &= 0 \text{ falls } |i-j| \geq 2 \\
        a_{i, i+1} &= a(\varphi_i, \varphi_{i+1}) = \int_{x_i}^{x_{i+1}} \varphi_i'(x) \varphi_{i+1}'(x) \ \mathrm dx
        = h \frac 1 h \left( - \frac 1 h \right) = - \frac 1 h \\
    a_{i+1, i} &= a_{i, i+1}
\end{align*}

\vspace{1em} Bestimme $\vec \ell = (\ell_i)_i$:
\[
    \ell_i = \ell(\varphi_i) = \int_{x_{i-1}}^{x_i} f(x) \varphi_i(x) \ \mathrm dx
\]
Um das Integral auszuwerten verwendet man im Allgemeinen eine Quadraturformel.

\subsubsection*{Finite Elemente Grundalgorithmus}

Betrachte
\begin{align*}
    - \left( c(x) u'(x) \right)' &= f(x) \text{ für } x \in \Omega = (0,1) \\
    u(0) = u(1) &= 0
\end{align*}

Variationsforumulierung: Finde $u \in H_0^1(\Omega) = V$ so, dass
\[
    \int_0^1 c(x) u'(x) v'(x) \ \mathrm dx = \int_0^1 f(x) v(x) \ \mathrm dx
\]
für alle $v \in V$.
Sei $\tau_h = \{ K_j \}_{j=1}^N$, $K_j = (x_{j-1}, x_j)$, $|x_j - x_{j-1}| \leq h$ für $j = 1, \ldots, N$ und
$\overline \Omega = \bigcup_{j=1}^N \overline K_j$.

Setze
\[
    V_h = \{ v \in C(\overline \Omega) \ : \ v(0) = v(1) = 0, \ v_{K_j} \in \mathcal P^k(K_j) \} \subset V, \; k \geq 1
\]

Da $\dim( \mathcal P^k(K_j)) = k+1$, wählen wir auf jedem Element $k+1$ Basisfunktionen.

Wie zuvor, sei 
\[
    u_h(x) = \sum_{j=1}^N u_j \varphi_j(x) \ .
\]
Dann ist
\[
    u_h(x_i) = \sum_{j=1}^N u_j \varphi_j(x_i) = \sum_{j=1}^N u_j \delta_{ij} = u_i .
\]

Für $i,j = 1, \ldots, N$, $a_{ij} = a(\varphi_i, \varphi_j)$, $\mat A = (a_{ij})$, $\ell_i = (f, \varphi_i)$, löse
das lineare Gleichungssystem $\mat A \vec u = \vec \ell$.

$\mat A$ heisst Steifigkeitsmatrix, $\vec \ell$ heisst Lastvektor.

Diese Vorgehensweise ist algorithmisch ungeschickt, weil sie basisfunktionsorientiert ist. Es ist besser, sich an den 
Elementen zu orientieren.

\vspace{1em}
\textbf{Idee:} Zerlege $a(u,v)$ und $\ell(v)$ in Elementbeiträge:
\begin{align*}
    a(u,v) &= \sum_{K \in \tau_h} a_K(u,v) \\
    a_K(u,v) &= \int_K c(x) u'(x) v'(x) \ \mathrm dx \\
    \ell(v) &= \sum_{K \in \tau_h} \ell_K(v) \\
    \ell_K(v) &= \int_K f(x) v(x) \ \mathrm dx
\end{align*}

Die Elementbeiträge können unabhängig voneinander (und parallel) ausgerechnet werden.

Dazu transformiere jedes $K_j \in \tau_h$ auf ein Referenzelement $\hat K = (-1, 1)$ via der Elementabbildung
\begin{align*}
    F_{K_j}: \ \hat K &\to K_j \\
    \xi &\mapsto m_j + \frac{\xi h_j}{2}
\end{align*}
wobei $m_j = \displaystyle \frac{x_j + x_{j-1}}{2}$ und $h_j = x_{j} - x_{j-1}$.

Sei $k=1$. Wir definieren
\begin{align*}
    N_0(\xi) &= {\varphi_{j-1}}_{|K_j}(F_{K_j}(\xi) ) \\
    N_1(\xi) &= {\varphi_j}_{|K_j}( F_{K_j}(\xi) )
\end{align*}

Dies sind Polynome in $\xi$.

Wir haben 
\[
    N_0(\xi) = \frac{1-\xi}{2} \text{ und } N_1(\xi) = \frac{1+\xi}{2} \ .
\]

Die $2 \times 2$ Elementsteifigkeitsmatrix $\mat A_K$, $K \in \tau_h$ ist definiert durch
\begin{align*}
    (A_K)_{ij} &= a_K \left( N_i(F_K^{-1}(x) ), N_j( F_K^{-1}(x)) \right) \\
        &= \int_K c(x) \left( \frac{\mathrm d }{\mathrm dx} N_i(F_K^{-1}(x)) \right) 
        \left( \frac{\mathrm d}{\mathrm dx} N_j(F_K^{-1}(x)) \right) \ \mathrm dx \\
        &= \int_{\hat K} c( F_K(\xi)) N_i'(\xi) N_j'(\xi) \cdot \left(  \frac{2}{h_j} \right)^2 \frac{h_j}{2} \ \mathrm dx \\
        &= \int_{\hat K} c( F_K(\xi)) N_i'(\xi) N_j'(\xi) \cdot   \frac{2}{h_j}  \ \mathrm dx \\
\end{align*}
für $i = 0,1$ und $j=0,1$. Das dritte Gleichheitszeichen folgt wegen $x = m_j + \frac{\xi h_j}{2}$ und 
$\mathrm dx = \frac{h_j}{2} \mathrm d \xi$. Für $c(x)$ allgemein müssen diese Integrale durch eine Quadraturformel
approximiert werden. Oft ist jedoch $c(x)$ stückweise konstant, d.h. $c_{|K} = c_K$. Dann ist 
\[
    \mat {A_K} = \frac{2}{h_K} c_K \mat{\hat A}, \text{ für } K \in \tau_h
\]
mit z.B.
\[
    \hat A_{00} = \int_{-1}^1 N_0'(\xi) N_0'(\xi) \ \mathrm d \xi = \frac 1 2 \ ,
\]
usw., also 
\[
    \mat{\hat A} = \frac 1 2 
    \begin{bmatrix}
        1 & -1 \\
        -1 & 1
    \end{bmatrix}
\]

Analog gehen wir beim Lastvektor vor:
\[
    (\ell_K)_i = \int_{\hat K} f( F_K(\xi)) N_i(\xi) \frac{h_K}{2} \ \mathrm d \xi \ , \; i = 0,1
\]
Falls $f_{|K} = f_K$, dann ist
\[
    (\ell_K)_i = \frac{h_K}{2} f_K( \hat \ell)_i = 1  \; \forall K \in \tau_h
\]
also
\[
\vec{\hat \ell} = \begin{bmatrix} 1 \\ 1 \end{bmatrix} \ .
\]

Schliesslich müssen aus $\mat{A_K}$ und $\vec{\ell_K}$ die globalen $\mat A$ und $\vec \ell$ zusammengesetzt werden 
(\textbf{Assemblierung}): 
\begin{align*}
    A_{jj} &= A_{K_j}(1,1) + A_{K_{j+1}}(0,0) \\
A_{j,j+1} &= A_{K_{j+1}} (0,1)
\end{align*}
    






