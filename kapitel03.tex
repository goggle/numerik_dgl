\chapter{Finite Elemente Verfahren für die Poisson-Gleichung}
\section{Variationsformulierung in 1D}

Betrachte
\begin{align*}
-u'' &= f \text{ in } \Omega = (0,1) \\
u(0) &= 0 \\ 
u'(1) &= 0
\end{align*}

Durch Integration erhalten wir:
\begin{align*}
u'(\xi) &= - \int_0^\xi f(t) \ \mathrm dt + u'(0) \\
u(x) &= - \int_0^x \int_0^\xi f(t) \ \mathrm dt \ \mathrm d\xi + x u'(0) + \underbrace{u(0)}_{=0} \\
0 &= u'(1) = - \int_0^1 f(t) \ \mathrm dt = u'(0)
\end{align*}

Wir erhalten also die Lösungsformel
\[
u(x) = x \int_0^1 f(t) \ \mathrm dt - \int_0^x \int_0^\xi f(t) \ \mathrm dt \ \mathrm d\xi \label{eq:lsg}
\]

\begin{bem}
Wir sehen: Falls $f \in C^k(\overline \Omega)$, dann ist $u \in C^{k+2} (\overline \Omega)$.
\end{bem}

\begin{defi}
$f: \Omega \to \mathbb R$ heisst \textbf{stückweise} $C^k$, wenn es endlich viele offene 
Teilgebiete $\Omega_i \subset \Omega$ gibt mit $f_{|\Omega_i} \in C^k(\overline \Omega_i)$.
Wir schreiben $f \in C^k_{stw}(\Omega)$.
\end{defi}

\begin{satz}
Sei $u$ gegeben durch (\ref{eq:lsg}) mit $f \in C^0_{stw}(\Omega)$. Dann ist
$u \in C^2_{stw} \cap C^1(\Omega)$.
\end{satz}
\begin{proof}
Klar.
\end{proof}

\begin{bem}
Die Lösungsformel (\ref{eq:lsg}) hat eine Sinn auf $\overline \Omega$, auch wenn die 
Differentialgleichung bei $x = s_i$ nicht gilt.
\end{bem}

\begin{idee}
Verlange (\ref{eq:lsg}) nicht punktweise sondern im Mittel. Dazu multipliziere 
$f$ mit einer Testfunktion $v \in C^1(\Omega)$, $v(0) = 0$ und integriere
partiell:
\[
\int_0^1 f(x) v(x) \ \mathrm dx = - \int_0^1 v(x) u''(x) \ \mathrm dx
    = \int_0^1 u'(x) v'(x) \ \mathrm dx - \underbrace{ \left[v(x) u'(x) \right]_0^1}_{=0}
\]
\end{idee}

Seien
\[
a(u,v) := \int_0^1 u'(x) v'(x) \ \mathrm dx
\]
und
\[
\ell(v) := \int_0^1 f(x) v(x) \ \mathrm dx \ .
\]

\textbf{Variationsformulierung:} Finde $u \in V$ ($V$ ist noch nicht näher definiert), so dass
\[
    a(u,v) = \ell(v), \; v \in V \label{eq:var} \tag{V}
\]

Für $u, v, w \in C^1(\overline \Omega)$, $\lambda \in \mathbb R$ gelten:
\begin{align*}
a(u+v, w) &= a(u,w) + a(v,w) \\
a(u, v+w) &= a(u,v) + a(u,w) \\
a(\lambda u, v) &= \lambda a(u,v) = a(u, \lambda v)
\end{align*}
Also ist $w$ eine Bilinearform.

\[
\ell(v + \lambda w) = \ell(v) + \lambda l(w)
\]
Also ist $\ell$ ein lineares Funktional.

Offen ist noch die Wahl von $V$.

\subsection{Sobolevräume in 1D}
\begin{idee}
Lineare Funktionalräume mit \textit{endlicher Energie}.
\[
H^k(\Omega) \supsetneqq C^k(\overline \Omega)
\]
\end{idee}

\begin{defi}
Sei $u \in C^0(\overline \Omega)$. Dann heisst $v: \Omega \to \mathbb R$ \textbf{schwache Ableitung} 
von $u$, falls
\[
\int_\Omega v \varphi \ \mathrm dx = - \int_\Omega u \varphi' \ \mathrm dx, \; \varphi \in C_0^1(\overline \Omega) \ .
\]
Analog heisst $v$ $k$-te schwache Ableitung von $u$, falls
\[
\int_\Omega v \varphi \ \mathrm dx = (-1)^k \int_\Omega u \varphi^{(k)} \ \mathrm dx, 
    \; \varphi \in C_0^k(\overline \Omega) \ ,
\]
wobei
\[
C_0^k(\overline \Omega) = \{ \varphi \in C^k(\overline \Omega) \ : \ 
    \varphi^{(j)}(0) = \varphi^{(j)}(1) = 0, \ 0 \leq j \leq k-1 \}
\]
\end{defi}

\begin{bspe}
    \begin{enumerate}
        \item Sei $u(x) := |x|$, $\Omega = (-1,1)$, dann ist
            \[
                v(x) = \begin{cases}
                    -1, & x < 0 \\
                    \text{nicht definiert}, & x = 0 \\
                    1, & x > 0
                \end{cases}
            \]
            die schwache Ableitung von $u$.
            Denn für $\varphi \in C_0^1(\overline \Omega)$ ist
            \[
                \int_{-1}^1 v(x) \varphi(x) \ \mathrm dx
                    = \int_{-1}^0 v(x) \varphi(x) \ \mathrm dx + \int_0^1 v(x) \varphi(x) \ \mathrm dx
                    = - \int_{-1}^0 \varphi(x) \ \mathrm dx + \int_0^1 \varphi(x) \ \mathrm dx
            \]
            und
            \begin{align*}
                - \int_{-1}^1 u(x) \varphi'(x) \ \mathrm dx
                &= \int_{-1}^0 x \varphi'(x) \mathrm dx - \int_0^1 x \varphi'(x) \ \mathrm dx \\
                &= - \int_{-1}^0 \varphi(x) \ \mathrm dx + \left[ x \varphi(x) \right]_{-1}^0
                + \int_0^1 \varphi(x) \ \mathrm dx - \left[ x \varphi(x) \right]_0^1 \ .
            \end{align*}
        \item Sei $u \in C^0(\overline \Omega) \cap C_{stw}^1(\Omega)$. Dann hat $u$ eine schwache Ableitung $u'$ gegeben
            durch
            \[
                u'(x) = \begin{cases}
                    \frac{\mathrm du}{\mathrm dx}(x), & x_{i-1} < x < s_i \\
                    \text{nicht definiert}, & x = s_i
                \end{cases}
            \]
        \item Falls $u \in C^1(\overline \Omega)$, d.h. $u$ ist differenzierbar im üblichen Sinn, dann ist die schwache
            Ableitung gleich der klassischen Ableitung.
    \end{enumerate}
\end{bspe}

\begin{konv}
Alle Ableitungen werden als schwache Ableitungen verstanden.
\end{konv}

\begin{defi}
    Sei $\Omega = (0,1)$. Dann ist
    \begin{align*}
        H^1(\Omega) &:= \{ u \in L^2(\Omega) \ : \ u' \in L^2(\Omega) \} \\
        H^k(\Omega) &:= \{ u \in L^2(\Omega) \ : \ u^{(j)} \in L^2(\Omega), \ 0 \leq j \leq k \} \\
        H^0(\Omega) &:= L^2(\Omega) := \left\{ u: \Omega \to \mathbb R \text{ messbar } : \ \int_\Omega |u|^2 \ \mathrm dx < \infty \right\} \\
        \Vert u \Vert_{L^2(\Omega)}^2 &= \int_\Omega | u(x) |^2 \ \mathrm dx
    \end{align*}
\end{defi}

\begin{bem}
    Es gilt:
    \[
        L^2(\Omega) \supset H^1(\Omega) \supset H^2(\Omega) \supset \ldots \supset H^k(\Omega) \supset \ldots
    \]
\end{bem}

\begin{satz}
    $L^2(\Omega)$ ist ein Hilbertraum für das Skalarprodukt
    \[
        (u, v) := \int_\Omega u(x) v(x) \ \mathrm dx \ ,
    \]
    d.h. ein vollständiger linearer normierter Vektorraum (Banachraum) bezüglich der Norm
    \[
        \Vert u \Vert_{L^2(\Omega)} := \sqrt{ (u, u) } \ .
    \]
    Es gilt die Cauchy-Schwarzsche Ungleichung:
    \[
        | (u,v) | \leq \Vert u \Vert_{L^2(\Omega)} \Vert v \Vert_{L^2(\Omega)}
    \]
\end{satz}
\begin{proof}
Siehe Funktionalanalysis, Übungen.
\end{proof}

\begin{bem}
    \begin{enumerate}
        \item Man identifiziert $u, v \in L^2(\Omega)$ falls $u(x) = v(x)$ fast überall in $\Omega$, d.h.
            $\Vert u - v \Vert_{L^2(\Omega)} = 0$.
            Falls $u \in L^2(\Omega)$, macht es im Allgemeinen keinen Sinn von Punktwerten von $u(x)$ zu sprechen.
            Insbesondere sind $u(0)$ oder $u(1)$ nicht wohldefiniert.
        \item Sei $u \in L^2(\Omega)$.
            \[
                \int_\Omega u \varphi = 0, \; \forall \varphi \in C^\infty(\Omega) \quad \implies \quad u = 0 \text{ (im $L^2$-Sinn)}
            \]
    \end{enumerate}
\end{bem}

\begin{satz}
    Für $k \geq 0$ ist $H^k(\Omega)$ ein Hilbertraum für das Skalarprodukt
    \[
        (u, v) := \sum_{j=0}^k ( u^{(j)}, v^{(j)} )_{L^2(\Omega)}
    \]
    Die induzierte Norm ist gegeben durch
    \[
        \Vert u \Vert_{H^k(\Omega)} := (u, u)^{\frac 1 2} = \left( \sum_{j=0}^k \Vert u^{(j)} \Vert_{L^2(\Omega)}^2 \right)^{\frac 1 2}
    \]
    Weiter ist die $H^k(\Omega)$-Seminorm gegeben durch
    \[
        \vert u \vert_{H^k(\Omega)} := \Vert u^{(k)} \Vert_{L^2(\Omega)}
    \]
\end{satz}

\begin{bspe}
    \begin{enumerate}
        \item Wenn $u \in H^1(\Omega)$, dann ist
            \begin{align*}
                \Vert u \Vert_{H^1(\Omega)} &= \Vert u \Vert_1 = \sqrt{ \Vert u \Vert_{L^2}^2 + \Vert u' \Vert_{L^2}^2 } \\
                \vert u \vert_{H^1(\Omega)} &= \vert u \vert_1 = \Vert u' \Vert_{L^2(\Omega)}
            \end{align*}
            Wenn $\Omega = (0,1)$ und $u = 1$ auf $\Omega$, dann ist $|u|_1 = 0$ aber $\Vert u \Vert_{H^1(\Omega)} = 1$.
        \item $u(x) = |x - \frac 1 2 |^{- \frac 1 4} \in L^2(\Omega)$, aber $u(\frac 1 2) = \infty$.
    \end{enumerate}
\end{bspe}

\begin{satz}
    Sei $u \in H^1(\Omega)$, $\Omega = (0,1)$. Dann existiert ein $\tilde u \in C(\overline \Omega)$ mit $u(x) = \tilde u(x)$
    fast überall.
\end{satz}

\begin{bem}
    \begin{enumerate}
        \item Falls $u \in H^1(\Omega)$, macht es Sinn von Punkt- und Randwerten von $u$ zu sprechen. Damit sind immer die 
            Werte von $\tilde u$ gemeint ($u = \tilde u$ in $L^2$).
        \item $H^1(\Omega) \subset C^0(\overline \Omega)$
    \end{enumerate}
\end{bem}

\begin{bsp}
    Sei $\Omega = (-1, 1)$ und 
    \[
        u(x) := \begin{cases}
            0, & \text{falls } x < 0 \\
            1, & \text{falls } x > 0
        \end{cases}
    \]
    Dann ist $u \notin H^1(\Omega)$.

    Es gilt $u \in L^2(\Omega)$. Angenommen $u' \in L^2(\Omega)$, wobei $u'$ die schwache Ableitung von $u$ bezeichne.
    Dann gilt per Definition der schwachen Ableigung für jedes $\varphi \in C_0^1(\Omega)$:
    \[
        \int_{-1}^1 u' \varphi \ \mathrm  dx = - \int_{-1}^1 u \varphi' \ \mathrm dx = - \int_0^1 \varphi' \ \mathrm dx = \varphi(0)
    \]
    Aus der Cauchy-Schwarz-Ungleichung folgt:
    \[
        |\varphi(0)| = | (u', \varphi) | \leq \Vert u' \Vert_{L^2(\Omega)}  \Vert \varphi \Vert_{L^2(\Omega)} 
            \leq C \Vert \varphi \Vert_{L^2(\Omega)}
    \]
    Für jedes $C > 0$ gibt es aber ein $\varphi \in C_0^1(\Omega)$ mit
    \[
        | \varphi(0) | \geq C \Vert \varphi \Vert_{L^2(\Omega)} \ ,
    \]
    was ein Widerspruch darstellt.
\end{bsp}

\begin{satz}
    Für $m \geq 0$ gilt:
    $C^\infty(\Omega) \cap H^m(\Omega)$ liegt dicht in $H^m(\Omega)$.
\end{satz}

\begin{bem}
    Funktionen in $H^m(\Omega)$ lassen sich beliebig genau durch glatte Funktionen approximieren (bzgl. der 
    $\Vert \cdot \Vert_{H^m}$-Norm).
\end{bem}

\begin{defi}
    Die Vervollständigung von $C_0^\infty(\Omega)$ bzgl. der $\Vert \cdot \Vert_{H^m}$-Norm heisst
    $H_0^m(\Omega)$.
    \[
        C_0^\infty(\Omega) := \{ u \in C^\infty(\Omega) \ : \ \supp\{u\} \subset \subset \Omega \}
    \]
\end{defi}

\begin{bsp}
    Sei $\Omega = (0,1)$,  $H_0^1(\Omega) = \{ v \in H^1(\Omega) \ : \ v(0) = v(1) = 0 \}.$
    Sei $u(x) = x$ auf $\Omega$. Dann ist $u \in H^1(\Omega)$, aber $u \notin H_0^1(\Omega)$.

    Angenommen, $u \in H_0^1(\Omega)$. 
    Definiere $L: H^1(\Omega) \to \mathbb R$, $v \mapsto \int_0^1 v'(x) \ \mathrm dx$.
    
    Mit der Cauchy-Schwarz-Ungleichung folgt:
    \[
        |L(v)| = \left| \int_\Omega 1 \cdot v'(x) \ \mathrm dx \right|
            \leq \Vert v' \Vert_{L^2(\Omega)} \Vert 1 \Vert_{L^2(\Omega)}
            \leq \Vert v \Vert_H^1(\Omega)
    \]
    Also ist $L$ stetig.

    Sei $\{ \varphi_n \}_{n\geq 1} \subset C_0^\infty$ so, dass 
    $\Vert \varphi_n - u \Vert_{H^1(\Omega)} \to 0$ für $n \to \infty$.

    Es folgt
    \[ 
        |L(u) - L(\varphi_n) | = | L(u - \varphi_n) | \leq \Vert u - \varphi_n \Vert_{H^1(\Omega)} \to 0 \ ,
    \]
    d.h. $L(\varphi_n) \to L(u)$ für $n \to \infty$.
    Aber es gilt
    \[
        L(\varphi_n) = \int_0^1 \varphi'_n(x) \ \mathrm dx = 0 \text{ für } n \in \mathbb N \quad \text{und} \quad
        L(u) = \int_0^1 1 \ \mathrm dx = 1 \ ,
    \]
    also haben wir einen Widerspruch gefunden und damit ist $u \notin H_0^1(\Omega)$.
\end{bsp}

Wir können nun den Raum $V$ für \eqref{eq:var} festlegen:

Finde $u \in V := \{u \in H^1(\Omega) \ : \ u(0) = 0 \}$ so, dass
\[
    a(u,v) = \int_\Omega u' v' = \ell(v) = \int_\Omega f v \quad \forall v \in V
\]
